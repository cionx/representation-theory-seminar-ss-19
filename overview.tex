\documentclass[a4paper,10pt]{scrartcl}

\usepackage{style}

\titlehead{Graduate Seminar on Representation Theory \hfill Jendrik Stelzner \hfill July 4, 2019}
\title{More on the \\ Representation~Theory \\ of~$\GL(V)$}
\author{}
\date{}

\begin{document}

\maketitle

\vspace{-4em}

We’re working over the fixed ground field~$\Complex$.
We abbreviate~$\tensor_{\Complex}$ as~$\tensor$.

A \defemph{partition} of a natural number~$n \geq 0$ is a tuple of positive natural numbers~$\lambda = (\lambda_1, \dotsc, \lambda_t)$ with~$\lambda_1 \geq \dotsb \geq \lambda_t$ and~$\lambda_1 + \dotsb + \lambda_t = n$.
We write~$\size{\lambda} \defined n$ and denote by~$\length(\lambda) \defined t$ the \defemph{length} of~$\lambda$.
We observe that the length~$\ell(\lambda)$ coincides with the number of rows of the Young diagram of shape~$\lambda$.
We write~$\lambda \ispar n$ to mean that~$\lambda$ is a partition of~$n$.

In this talk we will continue to explain the following connections:
\[
  \begin{tikzcd}[sep = large]
    \begin{tabular}{c}
      represenation theory of \\
      the symmetric groups
    \end{tabular}
    \arrow[leftrightarrow]{r}
    \arrow{d}
    &
    \text{symmetric functions}
    \arrow{d}
    \\
    \begin{tabular}{c}
      representation theory \\
      of~$\GL(V)$
    \end{tabular}
    \arrow[leftrightarrow]{r}
    &
    \text{symmetric polynomials}
  \end{tikzcd}
\]


\section{Recalling the last talk}

A \defemph{filling} of shape~$\lambda$ assigns to each box of the Young diagram of~$\lambda$ is positive natural number, or more generally elements of some set.

In the last talk we have seen that the isomorphism classes of polynomial representations of~$\GL(V)$ are indexed by the set of all partitions,~$\Par$.
For every partition~$\lambda$ with~$\lambda = (\lambda_1, \dotsc, \lambda_t)$ let
\[
  \widetilde{M}^\lambda(V)
  =
  \Exterior^{\tilde{\lambda}_1}(V) \tensor \dotsb \tensor \Exterior^{\tilde{\lambda}_n}(V) \,.
\]
where~$\tilde{\lambda}$ is the transposed of~$\lambda$.
The tensor factors~$\Exterior^{\tilde{\lambda}_j}(V)$ hence comes from the~{\howmanyth{$j$}} column of (the Young diagram of)~$\lambda$, which has height~$\tilde{\lambda}_j$.
Note that~$\widetilde{M}^\lambda(V) = 0$ if~$\length\lambda) > m$, i.e.\

\begin{example}
  The Young diagram of~$\lambda = (3,3,2,1)$ is given by
  \[
    \ydiagram{3,3,2,1}
  \]
  and therefore
  \[
    \ytableausetup{boxsize=0.7em}
    \widetilde{M}^{\lambda}(V)
    =
    \Exterior^4(V) \tensor \Exterior^3(V) \tensor \Exterior^2(V) \,.
    \ytableausetup{boxsize=normal}
  \]
\end{example}

If~$T$ is a filling of shape~$\lambda$ with vectors~$v_1, \dotsc, v_n \in V$ then we get an associated element~$\tilde{v}_T$ of~$\widetilde{M}^{\lambda}(V)$.

\begin{example}
  For
  \[
    T
    =
    \begin{ytableau}
      v_1 & v_4 \\
      v_2 & v_5 \\
      v_3
    \end{ytableau}
  \]
  the associated element is given by
  \[
    \tilde{v}_T
    =
    (v_1 \wedge v_2 \wedge v_3) \tensor (v_4 \wedge v_5) \,.
  \]
\end{example}

The submodule~$Q^{\lambda}(V)$ of~$\widetilde{M}^{\lambda}(V)$ is by definition generated by all differences
\begin{equation}
  \label{generators for Q of GL}
  \tilde{v}_T - \sum_{T'} \tilde{v}_{T'}
\end{equation}
where we run through all fillings~$T'$ of~$\lambda$ that arise as follows:
We fix columns~$i$ and~$j$ of~$T$ with~$i < j$ and a fix a set of entries~$Y$ of the~{\howmanyth{$j$}} column.
We take any subset~$X$ of the~{\howmanyth{$i$}} column of~$T$.
Then we interchange the entries of~$X$ and~$Y$ while maintaining their vertical orders.
\begin{example}
  For~$\lambda = (2,2,2)$ we choose for the filling
  \[
    T
    =
    \begin{ytableau}
      v_1 & v_4 \\
      v_2 & v_5 \\
      v_3 & v_6
    \end{ytableau}
  \]
  the first two entries of the second column, so that~$i = 1$ and~$j = 2$.
  We then get the following resulting fillings:
  \[
    \begin{ytableau}
      v_1 & *(gray) v_4 \\
      v_2 & *(gray) v_5 \\
      v_3 &         v_6
    \end{ytableau}
    \longto
    \begin{ytableau}
      *(red) v_1 & *(gray) v_4 \\
      *(red) v_2 & *(gray) v_5 \\
             v_3 &         v_6
    \end{ytableau} \,,\,
    \begin{ytableau}
      *(red) v_1 & *(gray) v_4 \\
             v_2 & *(gray) v_5 \\
      *(red) v_3 &         v_6
    \end{ytableau} \,,\,
    \begin{ytableau}
             v_1 & *(gray) v_4 \\
      *(red) v_2 & *(gray) v_5 \\
      *(red) v_3 &         v_6
    \end{ytableau}
    \longto
    \underbrace{
    \begin{ytableau}
      v_4 & v_1 \\
      v_5 & v_2 \\
      v_3 & v_6
    \end{ytableau}
    }_{\defines T'_1}
    \, , \,
    \underbrace{
    \begin{ytableau}
      v_4 & v_1 \\
      v_2 & v_3 \\
      v_5 & v_6
    \end{ytableau}
    }_{\defines T'_2}
    \, , \,
    \underbrace{
    \begin{ytableau}
      v_1 & v_2 \\
      v_4 & v_3 \\
      v_5 & v_6
    \end{ytableau}
    }_{\defines T'_3}
  \]
  We hence quotient out the relation
  \[
    \tilde{v}_T - \tilde{v}_{T'_1} - \tilde{v}_{T'_2} - \tilde{v}_{T'_3} \,.
  \]
\end{example}

The irreducible polynomial representation of~$\GL(V)$ associated to a partition~$\lambda$ is now given by
\[
  S^{\lambda}(V)
  =
  \widetilde{M}^{\lambda}(V)/Q^{\lambda}(V)
\]

\begin{example}
  \leavevmode
  \begin{enumerate}
    \item
      For~$\lambda = (1, \dotsc, 1) \ispar n$ we have~$S^{\lambda}(V) = \Exterior^n(V)$.
    \item
      For~$\lambda = (n)$ we have~$S^{\lambda}(V) = \Symm^n(V)$.
  \end{enumerate}
\end{example}

\begin{remark}
  The submodule~$Q^{\lambda}(V)$ is already generated by those generators~\eqref{generators for Q of GL} in whose construction the columns~$i$ and~$j$ are adjacent, i.e.\ such that~$j = i+1$.
\end{remark}

The polynomial~{\representations{$\GL(V)$}} can also be described by their characters:
Using a basis~$e_1, \dotsc, e_d$ of~$V$ we may identify~$\GL(V)$ with~$\GL_m(\Complex)$.
For any polynomial representation~$\rho \colon \GL(V) \to \GL(W)$ the weight space~$W_\beta$ with~$\beta = (\beta_1, \dotsc, \beta_m) \in \Natural^m$ is given by
\[
  W_\beta
  =
  \{
    w \in W
  \suchthat
    \text{$\diag(x_1, \dotsc, x_d).w = x_1^{\beta_1} \dotsm x_m^{\beta_m} w$ for all~$x_1, \dotsc, x_m \in \Complex^{\times}$}
  \} \,.
\]
The representation~$W$ decomposes into weight spaces in the sense that~$W = \bigoplus_{\beta \in \Natural^d} W_\beta$ and its character is the polynomial
\[
  \Char(\Schur(x_1, \dotsc, x_m))
  =
  \tr(\rho(\diag(x_1, \dotsc, x_m)))
  =
  \sum_{\beta \in \Integer^m} \dim(W_\beta) x^\beta
  \in
  \Integer[x_1, \dotsc, x_m] \,.
\]
The character~$\Char(W)$ is actually a symmetric polynomial, and the representation~$W$ is (up to isomorphism) uniquely determined by its character.

\begin{example}
  \leavevmode
  \begin{enumerate}
    \item
      The basis~$e_{i_1} \wedge \dotsb \wedge e_{i_n}$ of~$\Exterior^n(V)$ with~$i_1 < \dotsb < i_n$ consists of weight vectors, and its character is
      \[
        \Char(\Exterior^n(V))(x_1, \dotsc, x_m)
        =
        \sum_{1 \leq i_1 < \dotsb < i_n \leq m} x_{i_1} \dotsb x_{i_n}
        =
        e_n(x_1, \dotsc, x_m) \,,
      \]
      the~{\howmanyth{$n$}} elementary symmetric polynomial in~$m$ variables.
    \item
      We see similarly that
      \[
        \Char(\Symm^n(V))(x_1, \dotsc, x_m)
        =
        \sum_{1 \leq i_1 \leq \dotsb \leq i_n \leq m} x_{i_1} \dotsb x_{i_n}
        =
        h_n(x_1, \dotsc, x_m) \,,
      \]
      the~{\howmanyth{$n$}} complete homogeneous symmetric polynomial in~$m$ variables.
  \end{enumerate}
\end{example}





\section{From~$\symm_n$ to~$\GL(V)$}

We denote for any group~$G$ by~$\trivrep_G$ the trivial representation of~$G$.
We denote for~$n \geq 1$ by~$\signrep_n$ the sign representation of the symmetric group~$\symm_n$.

For every~$n \geq 0$ the tensor power~$V^{\tensor n}$ is again a~{\representation{$\GL(V)$}}, and it is also a right~{\representation{$\symm_n$}} via
\[
  (e_1 \tensor \dotsb \tensor e_n) . \sigma
  =
  e_{\sigma(1)} \tensor \dotsb \tensor e_{\sigma(n)} \,.
\]
It follows that for every~{\representation{$\symm_n$}}~$E$ we get a new~{\representation{$\GL(V)$}}
\[
  \Schur(E)
  \defined
  V^{\tensor n} \tensor_{\Complex[\symm_n]} E
\]
This construction results an exact (additive) functor
\[
  \Schur
  \colon
  \rep{\symm_n}
  \to
  \prep{\GL(V)} \,,
\]
the \defemph{Schur functor}.

\begin{example}
  \leavevmode
  \begin{enumerate}
    \item
      We have that~$\trivrep_{\symm_n} \cong \Complex[\symm_n]/\gen{\sigma - 1 \suchthat \sigma \in \symm_n}_{\Complex}$ and thus
      \begin{align*}
        \Schur( \trivrep_{\symm_n} )
        &\cong
        V^{\tensor n} \tensor_{\Complex[\symm_n]} \Complex[\symm_n]/\gen{\sigma - 1 \suchthat \sigma \in \symm_n}_{\Complex}
        \\
        &\cong
        V^{\tensor n}/\gen{x - x.\sigma \suchthat t \in V^{\tensor n}}_{\Complex}
        \\
        &\cong
        \Symm^n(V) \,.
      \end{align*}
    \item
      We find similarly that~$\Schur( \signrep_n ) \cong \Exterior^n(V)$.
  \end{enumerate}
\end{example}

If~$V_1, \dotsc, V_t$ are representations of the groups~$\symm_{n_1}, \dotsc, \symm_{n_t}$ then for~$n = n_1 + \dotsb + n_t$,
\[
  V_1 \circ \dotsb \circ V_t
  \defined
  \Complex[\symm_n]
  \tensor_{\Complex[\symm_{n_1} \times \dotsb \times \symm_{n_t}]}
  (V_1 \tensor \dotsb \tensor V_t)
\]
is a representation of~$\symm_n$.

\begin{lemma}
  \label{properties of circle product}
  Let~$V_i$,~$V$ and~$W$ be representations of symmetric groups.
  Then
  \begin{enumerate}
    \item
      $(V_1 \circ \dotsb \circ V_s) \circ \dotsb \circ (W_1 \circ \dotsb \circ W_t) \cong V_1 \circ \dotsb \circ W_t$,
    \item
      $V \circ W \cong W \circ V$.
  \end{enumerate}
\end{lemma}

\begin{lemma}
  \label{multiplicativity of schur functor}
  If~$E$ and~$F$ are representations of symmetric groups~$\symm_a$ and~$\symm_b$ then
  \[
    \Schur(E \circ F) \cong \Schur(V) \tensor \Schur(W) \,.
  \]
\end{lemma}

\begin{proof}
  We find that
  \begin{align*}
    \Schur(V \circ W)
    &=
    E^{\tensor (a+b)}
    \tensor_{\Complex[\symm_{a+b}]}
    \Complex[\symm_{a+b}]
    \tensor_{\Complex[\symm_a \times \symm_b]}
    (V \tensor W)
    \\
    &\cong
    E^{\tensor (a+b)}
    \tensor_{\Complex[\symm_a \times \symm_b]}
    (V \tensor W)
    \\
    &\cong
    (E^{\tensor a} \tensor E^{\tensor b})
    \tensor_{\Complex[\symm_a] \tensor \Complex[\symm_b]}
    (V \tensor W)
    \\
    &\cong
    (E^{\tensor a} \tensor_{\Complex[\symm_a]} V)
    \tensor
    (E^{\tensor b} \tensor_{\Complex[\symm_b]} W)
    \\
    &=
    \Schur(V) \tensor \Schur(W)
  \end{align*}
  as claimed.
\end{proof}





\section{Some representation theory of~$\symm_n$}

A \defemph{numbering} of a Young tableaux~$Y$ of shape~$\lambda \ispar n$ fills in the boxes of~$Y$ with the numbers~$1, \dotsc, n$.

\subsection{The representation $M^\lambda$}

Two numbering~$T$ and~$T'$ of a Young diagram of shape~$\lambda$ are \defemph{row-equivalent} if they have the same entries in each row.
A \defemph{row tabloid} is an equivalence class of row-equivalent numbering.
A row tabloid can be represented as follows:
\[
  \Tabloid{{1,2,3},{4,5,6},{7,8}}
  =
  \Tabloid{{2,1,3},{4,6,5},{8,7}}
\]
Every numbering~$T$ defines a tabloid~$[T]$.

The group~$\symm_n$ acts transitive on the set of numberings of shape~$\lambda \ispar n$, which induces a transitive group action on the set of row tabloids of shape~$\lambda$.
The \defemph{row group}~$R(T)$ of a numbering~$T$ is given by all permutations~$\sigma \in \symm_n$ that act row-wise on~$T$, i.e.\ the stabilizer of the associated tabloid~$[T]$.

It follows that~$\symm_n$ acts linearly in~$M^\lambda$, the free vector space on the set of Young~tabloids of shape~$\lambda$, and that for any numbering~$T$ of shape~$\lambda$,
\begin{align*}
  M^\lambda
  &\cong
  \Complex[\symm_n]/(p - 1 \suchthat p \in R(T))
  \\
  &\cong
  \Complex[\symm_n] \tensor_{\Complex[R(T)]} \Complex[R(T)]/(p - 1 \suchthat p \in R(T))
  \\
  &\cong
  \Complex[\symm_n] \tensor_{\Complex[R(T)]} \trivrep_{R(T)}
\end{align*}
If~$T$ is the \enquote{horizontal standard numbering}, e.g.
\[
  \begin{ytableau}
     1 &  2 & 3 & 4 & 5 \\
     6 &  7 & 8 & 9 \\
  \end{ytableau}
\]
for~$\lambda = (5,4)$, then~$R(T) = \symm_{\lambda_1} \times \dotsb \times \symm_{\lambda_t}$ where~$\lambda = (\lambda_1, \dotsc, \lambda_t)$ and thus
\begin{align*}
  M^\lambda
  &\cong
  \Complex[\symm_n] \tensor_{\Complex[R(T)]} \trivrep_{R(T)}
  \\
  &\cong
  \Complex[\symm_n]
  \tensor_{\Complex[\symm_{\lambda_1} \times \dotsb \times \symm_{\lambda_t}]}
  \trivrep_{\symm_{\lambda_1} \times \dotsb \times \symm_{\lambda_t}}
  \\
  &\cong
  \Complex[\symm_n]
  \tensor_{\Complex[\symm_{\lambda_1} \times \dotsb \times \symm_{\lambda_t}]}
  (\trivrep_{\symm_{\lambda_1}} \tensor \dotsb \tensor \trivrep_{\symm_{\lambda_t}})
  \\
  &=
  \trivrep_{\symm_{\lambda_1}} \circ \dotsb \circ \trivrep_{\symm_{\lambda_t}} \,.
\end{align*}
It follows that
\begin{align*}
  \Schur(M^\lambda)
  &\cong
  \Schur(\trivrep_{\symm_{\lambda_1}} \circ \dotsb \circ \trivrep_{\symm_{\lambda_t}})
  \\
  &\cong
  \Schur(\trivrep_{\symm_{\lambda_1}}) \tensor \dotsb \tensor \Schur(\trivrep_{\symm_{\lambda_t}})
  \\
  &\cong
  \Symm^{\lambda_1}(E) \tensor \dotsb \tensor \Symm^{\lambda_t}(E) \,,
\end{align*}
or alternatively that
\begin{align*}
  \Schur(M^\lambda)
  &=
  V^{\tensor n} \tensor_{\Complex[\symm_n]} \Complex[\symm_n]/(p - 1 \suchthat p \in R(T))
  \\
  &\cong
  V^{\tensor n}/(x - x.p \suchthat p \in R(T))
  \\
  &\cong
  V^{\tensor n}/(x - x.p \suchthat p \in \symm_{\lambda_1} \times \dotsb \times \symm_{\lambda_t})
  \\
  &\cong
  \Symm^{\lambda_1}(E) \otimes \dotsb \otimes \Symm^{\lambda_t}(E) \,.
\end{align*}



\subsection{The representation~$\widetilde{M}^\lambda$}

We can alter the construction of~$M^\lambda$ in two ways:
Working with column instead of rows and introducing an alternating sign:

We denote by~$C(T)$ the column group of a numbering~$T$, i.e.\ all permutations that act column-wise on~$T$.
We let~$\widetilde{M}$ be the free vector generated by the numbering~$T$ of shape~$\lambda \ispar n$ subject to the relations~$T = \sign(\sigma) \sigma.T$ for~$\sigma \in C(T)$.
The resulting generators~$[T]$ of~$\widetilde{M}^\lambda$ may be visualized as follows:
\[
  \ColumnTabloid{{1,4},{2,5},{3}}
  =
  - \ColumnTabloid{{2,4},{1,5},{3}}
  =
  \ColumnTabloid{{2,5},{1,4},{3}}
\]
The group~$\symm_n$ acts on~$\widetilde{M}^\lambda$ via~$\sigma.[T] = [\sigma.T]$ for any numbering~$T$ of shape~$\lambda$, and
\begin{align*}
  \widetilde{M}^\lambda
  &\cong
  \Complex[\symm_n]/(q - \sign(q)1 \suchthat q \in C(T))
  \\
  &\cong
  \signrep_{\lambda_1} \circ \dotsb \circ \signrep_{\lambda_t} \,.
\end{align*}
It follows that
\[
  \Schur\bigl( \widetilde{M}^\lambda \bigr)
  \cong
  \Exterior^{\lambda_1}(E) \tensor \dotsb \tensor \Exterior^{\lambda_t}(E)
  =
  \widetilde{M}^{\lambda}(E) \,.
\]

\begin{example}
  \label{isomorphism in explicit}
  Let~$\lambda = (2,2,1)$.
  Let~$v_1, \dotsc, v_5 \in E$ and consider an arbitrary filling
  \[
    \ytableausetup{boxsize=1.8em}
    T
    =
    \begin{ytableau}
      \scriptstyle \sigma(1) & \scriptstyle \sigma(4) \\
      \scriptstyle \sigma(2) & \scriptstyle \sigma(5) \\
      \scriptstyle \sigma(3)
    \end{ytableau}
    \ytableausetup{boxsize=normal}
  \]
  of shape~$\lambda$, where~$\sigma \in \symm_5$.
  In~$\Schur(\widetilde{M}^\lambda) = V^{\tensor n} \tensor_{\Complex[\symm_n]} \widetilde{M}^\lambda$ we get an element
  \begin{align*}
    (v_1 \tensor \dotsb \tensor v_5) \tensor [T]
    &=
    (v_1 \tensor \dotsb \tensor v_5) \tensor [\sigma.T_0]
    \\
    &=
    ((v_1 \tensor \dotsb \tensor v_5).\sigma) \tensor [T_0]
    \\
    &=
    v_{\sigma(1)} \tensor \dotsb \tensor v_{\sigma(5)} \tensor [T_0]
  \end{align*}
  where
  \[
    T_0
    =
    \begin{ytableau}
      1 & 4 \\
      2 & 5 \\
      3
    \end{ytableau}
  \]
  is the \enquote{vertical standard numbering}.
  By identifying~$\widetilde{M}^\lambda$ with~$\Complex[\symm_n]/(q - 1 \suchthat q \in C(T))$ via~$[\tau] \mapsto \tau.[T_0] = [\tau.T_0]$ and using that~$C(T) = \symm_3 \times \symm_2$ we find that the corresponding element of~$\widetilde{M}^{\lambda}(V) = \Exterior^3(V) \tensor \Exterior^2(V)$ is given by
  \[
    (v_{\sigma(1)} \wedge v_{\sigma(2)} \wedge v_{\sigma(3)})  \tensor (v_{\sigma(4)} \wedge v_{\sigma(5)})
  \]
  Note that this is precisely the element~$\tilde{v}_{\widetilde{T}}$ for the filling~$\tilde{T}$ given as follows:
  \[
    \ytableausetup{boxsize=2em}
    \widetilde{T}
    =
    \begin{ytableau}
      \scriptstyle v_{\sigma(1)} & \scriptstyle v_{\sigma(4)} \\
      \scriptstyle v_{\sigma(2)} & \scriptstyle v_{\sigma(5)} \\
      \scriptstyle v_{\sigma(3)}
    \end{ytableau}
    \ytableausetup{boxsize=normal}
  \]
\end{example}




\subsection{Specht modules}

The irreducible representations of the symmetric group~$\symm_n$ can be indexed by the partitions~$\lambda \in \Par(n)$ and constructed as follows:

If~$T$ is any numbering of shape~$\lambda$ then its \defemph{Young centralizer} is the element
\[
  c_T
  \defined
  \sum_{q \in C(T)} \sign(q) q
  \in
  \Complex[\symm_n] \,.
\]
The \defemph{Specht module}~$S^{\lambda}$ is the linear subspace of~$M^{\lambda}$ spanned by~$v_T \defined c_T \cdot [T]$ as~$T$ runs through the numbering of shape~$\lambda$.
It holds that~$\sigma \cdot v_T = v_{\sigma \cdot T}$ for every~$\sigma \in \symm_n$ whence~$S^\lambda$ is a subrepresentation of~$M^\lambda$.

\begin{theorem}[Classification of irreducible representations of~$\symm_n$]
  \label{irreps of sn}
  \leavevmode
  \begin{enumerate}
    \item
      The representations~$S^\lambda$ with~$\lambda \ispar n$ are pairwise non-isomorphic irreducible representations of the group~$\symm_n$, and every irreducible representation of~$\symm_n$ is of this form.
    \item
      The elements~$v_T$, where~$T$ runs through the standard Young tableaux of shape~$\lambda$, form a basis of~$S^\lambda$.
      Hence
      \[
        \dim S^\lambda
        =
        \text{number of standard Young tableaux of shape~$\lambda$} \,.
      \]
  \end{enumerate}
\end{theorem}

One can also construct the Spect module~$S^{\lambda}$ as as a quotient of the representation~$\widetilde{M}^\lambda$:
The map
\[
  \widetilde{M}^\lambda
  \to
  S^\lambda
  \quad
  [T]
  \mapsto
  v_T
\]
is a well-defined surjective homomorphism of~{\representations{$\symm_n$}}.
Its kernel~$Q^\lambda$ is generated by the differences
\[
  [T] - \sum_{T'} [T']
\]
where~$[T']$ runs through all numberings of~$\lambda$ that arise from~$T$ in the following way:
We fix a column~$j = 2, \dotsc, \lambda_1$ and fix a set~$Y$ of entries the~{\howmanyth{$j$}} of~$T$.
Then fory any equinumerous subset~$X$ of the~{\howmanyth{$(j-1)$}} column of~$T$ we get such a~$T'$ by interchanging the entries of~$X$ and~$Y$ while presvering their vertical ordering.
We find the following:

\begin{theorem}
  For any partition~$\lambda$,
  \[
    \Schur(S^\lambda) \cong S^{\lambda}(V) \,.
  \]
\end{theorem}

\begin{proof}
  By applying the (exact) functor~$\Schur(-)$ to the short exact sequence
  \[
    0
    \to
    Q^\lambda
    \to
    \widetilde{M}^\lambda
    \to
    S^\lambda
    \to
    0
  \]
  we get the short exact sequence
  \[
    0
    \to
    \Schur(Q^\lambda)
    \to
    \Schur(\widetilde{M}^\lambda)
    \to
    \Schur(S^\lambda)
    \to
    0 \,.
  \]
  We have previously seen that~$\Schur(\widetilde{M}^\lambda) \cong \widetilde{M}^\lambda(V)$.
  We see from the calculations showcased in \cref{isomorphism in explicit} and the explicit description of~$Q^\lambda$ and~$Q^{\lambda}(V)$ that the image of~$\Schur(Q^\lambda)$ in~$\Schur(\widetilde{M}^\lambda)$ corresponds precisely to~$Q^{\lambda}(V)$.
  Hence
  \[
    \Schur(S^\lambda)
    \cong
    \Schur(\widetilde{M}^\lambda)/\Schur(Q^\lambda)
    \cong
    \widetilde{M}^{\lambda}(V)/Q^{\lambda}(V)
    \cong
    S^{\lambda}(V) \,.
  \]
  This proves the assertion.
\end{proof}


\begin{remark}
  \label{remark on schur functor}
  \leavevmode
  \begin{enumerate}
    \item
      One can similarly construct the irreducible representations of~$\symm_n$ as subrepresentations~$\widetilde{S}^\lambda$ of~$M^\lambda$.
      Then~$\widetilde{S}^\lambda$ is a quotient of~$M^\lambda$ and the compositions~$S^\lambda \to M^\lambda \to \widetilde{S}^\lambda$ and~$\widetilde{S}^\lambda \to \widetilde{M}^\lambda \to S^\lambda$ are isomorphisms.
      Hence~$S^\lambda \cong \tilde{S}^\lambda$.
    \item
      It then follows that~$\Schur(\widetilde{S}^\lambda) \cong \Schur(S^\lambda) \cong S^{\lambda}(V)$.
      If~$\lambda = (\lambda_1, \dotsc, \lambda_t)$ then it follows from~$\Schur(M^\lambda) \cong \Symm^{\lambda_1}(E) \tensor \dotsb \tensor \Symm^{\lambda_t}(E)$ similarly to above that the Schur modules~$S^{\lambda}(V)$ can also be constructed as quotients of~$\Symm^{\lambda_1}(E) \tensor \dotsb \Symm^{\lambda_t}(E)$.
    \item
      \label{kernel of schur functor}
      If~$\length(\lambda) > m$, i.e.\ if the Young diagram of~$\lambda$ constrists of more than~$m$ rows, then~$\Schur(S^\lambda) = S^\lambda(V) = 0$.
      If~$\length(\lambda) \leq m$ then~$\Schur(S^\lambda) = S^\lambda(V)$ is again an irreducible representation.
  \end{enumerate}
\end{remark}


\begin{corollary}
  For any~$n \geq 0$,~$V^{\tensor n} \cong \bigoplus_{\lambda \ispar n} S^{\lambda}(V)^{\oplus f_\lambda}$ where~$f_\lambda$ denotes the number of standard Young tableaux of shape~$\lambda$.
\end{corollary}


\begin{proof}
  We find with \cref{irreps of sn} that
  \[
    \Complex[\symm_n]
    \cong
    \bigoplus_{\lambda \ispar n} (S^\lambda)^{\oplus f^\lambda} \,.
  \]
  with
  \begin{align*}
    f_\lambda
    &=
    \text{multiplicity of~$S_\lambda$ in~$\Complex[\symm_n]$}
    \\
    &=
    \text{dimension of~$S_\lambda$}
    \\
    &=
    \text{number of standard Young tableaux of shape~$\lambda$}
  \end{align*}
  where the second equality follows from the Artin--Wedderburn theorem.
  It follows that
  \begin{align*}
    V^{\tensor n}
    &=
    V^{\tensor n} \tensor_{\Complex[\symm_n]} \Complex[\symm_n]
    \\
    &\cong
    \Schur(\Complex[\symm_n])
    \\
    &\cong
    \Schur\left( \bigoplus_{\lambda \ispar n} (S^\lambda)^{\oplus f^\lambda} \right)
    \\
    &\cong
    \bigoplus_{\lambda \ispar n} \Schur(S^\lambda)^{\oplus f^\lambda}
    \\
    &\cong
    \bigoplus_{\lambda \ispar n} S^{\lambda}(V)^{\oplus f^\lambda}
  \end{align*}
  as claimed.
\end{proof}


\begin{example}
  The partitions of~$n = 2$ are~$\lambda_1 = (1,1)$ and~$\lambda_2 = (2)$.
  Each of those partitions admits precisely one standard Young tableau:
  \[
    \begin{ytableau}
      1 & 2
    \end{ytableau}
    \qquad
    \begin{ytableau}
      1 \\
      2
    \end{ytableau}
  \]
  It follows that
  \[
    V \tensor V
    =
    V^{\tensor 2}
    \cong
    S^{\lambda_1}(V)^{\oplus 1}
    \oplus
    S^{\lambda_2}(V)^{\oplus 1}
    \cong
    \Symm^2(V) \oplus \Exterior^2(V) \,.
  \]
\end{example}





\section{Translation into rings}



\subsection{The Grothendieck ring of~$\GL(V)$}

Recall that the Grothendieck groups of an abelian category~$\Acat$ is generated by the isomorphism classes of objects of~$\Acat$ subject to the relation~$[B] = [A] + [C]$ for every short exact sequence~$0 \to A \to B \to C \to 0$.
Then in particular~$[A] + [B] = [A \oplus B]$ for all~$A, B \in \Acat$.
If the abelian category~$\Acat$ is semisimple, i.e.\ if every short exact sequence in~$\Acat$ splits, then these conditions already suffice.

Let~$\Acat$ be the semisimple category of finite dimensional polynomial representations of~$\GL(V)$.
We denote the Grothendieck group of~$\Acat$ by~$\K_0(\GL(V))$.
This becomes a ring when endowed with the multiplication
\[
  [V] \cdot [W]
  =
  [V \tensor W] \,.
\]
It follows from last week’s classification of irreducible finite dimensional polynomial~{\representations{$\GL(V)$}} that~$\K_0(\GL(V))$ has a basis given by~$[S^\lambda(V)]$ where~$\lambda$ ranges through all partitions with~$\length\lambda) \leq m$.



\subsection{The representation ring of~$\symm_n$}

For every~$n \geq 0$ let~$R_n$ be the Grothendieck group of~$\rep{S_n}$, the category of finite dimensional~{\representations{$S_n$}}.
We can define on~$R \defined \bigoplus_{n \geq 0} R_n$ a multiplication via
\[
  [E] \cdot [F]
  =
  [E \circ F] \,.
\]
Note that~$[\trivrep_{S_0}]$ is multiplicative neutral and that this multiplicaton is associative and commutative by \cref{properties of circle product}.
The multiplication is also distributive and~$R_i R_j \subseteq R_{i+j}$ for all~$i,j \geq 0$.
Hence~$R$ becomes a graded ring.
The category~$\rep{S_n}$ is semisimple by Maschke’s theorem and it follows from \cref{irreps of sn} that~$R_n$ has as a~{\basis{$\Integer$}} given by~$[S^\lambda]$ for~$\lambda \ispar n$.
Hence~$R$ has a~{\basis{$\Integer$}} given by~$[S^\lambda]$ where~$\lambda$ ranges through all partitions.

For every~$n \geq 0$ the additivity of the Schur functor~$\Schur \colon \rep{S_n} \to \Acat$ gives a group homomorphism~$R_n \to \K_0(\GL(V))$, which together give a group homomorphism~$R \to \K_0(\GL(V))$.
It follows from \cref{multiplicativity of schur functor} that this is a ring homomorphism.
By abuse of notation we denote this homomorphism again by~$\Schur$.

We have argued in part~\ref{kernel of schur functor} of \cref{remark on schur functor} that the kernel of~$\Schur \colon R \to \K_0(\GL(V))$ is spanned by those~$[S^\lambda]$ with~$\length\lambda) > m$, whereas all other basis vector~$[S^\lambda]$ with~$\length\lambda) \leq m$ are mapped bijectively onto the basis~$[S^\lambda(V)]$ of~$\K_0(\GL(V))$.


\subsection{The ring of symmetric functions}

For every~$k \geq 0$ let~$\Lambda(k)$ denote the ring of symmetric polynomials, a subring of~$\Integer[x_1, \dotsc, x_k]$.

When dealing with symmetric polynomials it often happens that the number of variables,~$k$, does not matter:
One has a family~$(f_k)_{k \geq 0}$ of symmetric polynomials~$f_k \in \Lambda(k)$ such that~$f_k(x_1, \dotsc, x_{k-1}, 0) = f_{k-1}(x_1, \dotsc, x_{k-1})$ for every~$k \geq 1$, and an identity involving~$f_n$ which reduces for~$x_k \to 0$ to the identity involving~$f_{k-1}$.

\begin{example}
  The elementary symmetric polynomials~$e_n(x_1, \dotsc, x_k)$ and the completely symmetric polynomials~$h_n(x_1, \dotsc, x_k)$ are for every~$k \geq 1$ related by the formula
  \[
    \sum_{i=0}^s (-1)^i h_{s-i}(x_1, \dotsc, x_k) e_i(x_1, \dotsc, x_k) = 0 \,.
  \]
  for every~$s \geq 0$.
  This identity in~$k-1$ variables follows from the one in~$k$ variables by setting~$x_k \to 0$.
\end{example}

To formalize this phenomenon we introduce the \defemph{ring of symmetric functions}~$\Lambda$:
For every degree~$n \geq 0$ we set
\begin{align*}
  \Lambda_n
  &\defined
  \left\{
    (f_k)_{k \geq 0}
  \suchthat*
    \begin{tabular}{c}
      $f_k \in \Lambda_n(k)$ with \\
      $f_k(x_1, \dotsc, x_{k-1}, 0) = f_{k-1}(x_1, \dotsc, x_{k-1})$ \\
      for every~$k \geq 1$
    \end{tabular}
  \right\}
  \\
  &=
  \lim( \Lambda_n(0) \from \Lambda_n(1) \from \Lambda_n(2) \from \Lambda_n(3) \from \dotsb)
\end{align*}
where~$\Lambda_n(k) \to \Lambda_n(k-1)$ is the group homomorphism given by~$x_k \to 0$.
We combine these groups into a graded ring~$\Lambda = \bigoplus_{n \geq 0} \Lambda_n$ with multiplication given by
\[
  (f_k)_{k \geq 0} \cdot (g_k)_{k \geq 0}
  \defined
  (f_k g_k)_{k \geq 0} \,.
\]

\begin{example}
  For every~$n \geq 0$ we have an element
  \[
    e_n
    \defined
    (e_n(), e_n(x_1), e_n(x_1, x_2), e_n(x_1, x_2, x_3), \dotsc)
    \in
    \Lambda_n
  \]
  and similarly elements~$h_n, p_n \in \Lambda_n$.
  We get for every partition~$\lambda = (\lambda_1, \dotsc, \lambda_t)$ an induced element
  \[
    e_{\lambda}
    \defined
    e_{\lambda_1} \dotsm e_{\lambda_t}
    \in
    \Lambda_{\size{\lambda}}
  \]
  and similarly elements~$h_{\lambda}, p_{\lambda}0\in \Lambda_{\size{\lambda}}$.
\end{example}

\begin{remark}
  \leavevmode
  \begin{enumerate}
    \item
      We have that
      \[
        \Lambda
        =
        \lim( \Lambda(0) \from \Lambda(1) \from \Lambda(2) \from \Lambda(3) \from \dotsb)
      \]
      in the category of graded rings.
    \item
      The elements of the ring~$\Lambda$ are not functions, despite its name.
  \end{enumerate}
\end{remark}


\begin{example}[Schur polynomials  and Schur functions]
  Let~$\lambda$ be a partition.
  For every semistandard Young tableaux~$T$%
  \footnote{This means that~$T$ is a filling that is weakly increasing in each row but strictly increasing in each column}
  of shape~$\lambda$ with entries in~$\{1, \dotsc, k\}$ let
  \[
    x_T
    \defined
    \prod_{i \in T} x_i \,.
  \]
  The \defemph{Schur polynomial}~$s_\lambda(x_1, \dotsc, x_k)$ is defined as
  \[
    s_\lambda(x_1, \dotsc, x_k)
    =
    \sum_{T'} x_{T'}
    \in
    \Integer[x_1, \dotsc, x_n]
  \]
  where~$T'$ ranges through the semistandard Young tableaux of shape~$\lambda$ with entries in~$\{1, \dotsc, k\}$.
  
  If for example~$\lambda = (2,2)$ and~$m = 3$ then the semistandard Young tableaux are as follows:
  \[
    \begin{ytableau}
      1 & 1 \\
      2 & 2
    \end{ytableau}
    \qquad
    \begin{ytableau}
      1 & 1 \\
      2 & 3
    \end{ytableau}
    \qquad
    \begin{ytableau}
      1 & 1 \\
      3 & 3
    \end{ytableau}
    \qquad
    \begin{ytableau}
      1 & 2 \\
      2 & 3
    \end{ytableau}
    \qquad
    \begin{ytableau}
      1 & 2 \\
      3 & 3
    \end{ytableau}
    \qquad
    \begin{ytableau}
      2 & 2 \\
      3 & 3
    \end{ytableau}
  \]
  The Schur polynomial~$s_{(2,2)}(x_1, x_2, x_3)$ is therefore given by
  \[
    s_{(2,2)}(x_1, x_2, x_3)
    =
    x_1^2 x_2^2 + x_1^2 x_2 x_3 + x_1^2 x_3^2 + x_1 x_2^2 x_3 + x_1 x_2 x_3^2 + x_2^2 x_3^3 \,.
  \]
  We observe the following:
  \begin{enumerate}
    \item
      The Schur polynomial~$s_\lambda(x_1, \dotsc, x_k)$ is homogeneous of degree~$\size{\lambda}$.
    \item
      If~$\length(\lambda) > k$, i.e.\ if the Young diagram of~$\lambda$ has more than~$k$ rows, then the Schur polynomial~$s_\lambda(x_1, \dotsc, x_k)$ vanishes since there exist no semistandard Young tableaux of shape~$\lambda$ with entries in~$\{1, \dotsc, k\}$.
      (We don’t have enough entries to make the first column strictly increasing, which is required for a semistandard Young tableaux.)
    \item
      If~$e_1, \dotsc, e_k$ is a basis of~$V$ then the Specht~module~$S^\lambda(V)$ has a basis~$e_T$ where~$T$ ranges through the semistandard Young tableaux with entries in~$\{1, \dotsc, k\}$.
      Each~$e_T$ is a weight vector with corresponding weight~$x_T$.
      Hence
      \[
        s_\lambda(x_1, \dotsc, x_k)
        =
        \Char(S^\lambda(V)) \,.
      \]
      This shows in particular that that~$s_\lambda(x_1, \dotsc, x_k)$ is a symmetric polynomial.
      We also see again that~$s_\lambda(x_1, \dotsc, x_k) = 0$ if~$\length(\lambda) > k$.
    \item
      It holds that~$s_\lambda(x_1, \dotsc, x_{k-1}, 0) = s_\lambda(x_1, \dotsc, x_{k-1})$.
  \end{enumerate}
  We find that we get a well-defined element~$s_\lambda \in \Lambda_{\size{\lambda}}$, the \defemph{Schur function}.
\end{example}

We have for every~$k \geq 0$ a homomorphism of graded rings
\[
  \Lambda
  \to
  \Lambda(k) \,,
  \quad
  f
  \mapsto
  f(x_1, \dotsc, x_k)
\]
that assigns to~$f$ the entailed symmetric polynomial in~$k$ variables.
An equality~$f = g$ holds in~$\Lambda$ if and only if for every~$k \geq 0$ the equality~$f(x_1, \dotsc, x_k) = g(x_1, \dotsc, x_k)$ hold.

\begin{example}
  For every~$s \geq 0$ the equality~$\sum_{i=0}^s (-1)^i h_{s-i} e_i = 0$ holds in~$\Lambda$.
\end{example}

\begin{proposition}
  For every~$k \geq 0$ the ring of symmetric polynomials~$\Lambda(k)$ has the Schur polynomials~$s_\lambda(x_1, \dotsc, x_k)$ with~$\length(\lambda) \leq k$ as a basis.
\end{proposition}

\begin{proposition}
  \leavevmode
  \begin{enumerate}
    \item
      The symmetric functions~$e_1, e_2, \dotsc$ generate~$\Lambda$ and are algebraically independent.
    \item
      The monomials~$e_\lambda$ where~$\lambda$ ranges through all partitions is a~{\basis{$\Integer$}} of~$\Lambda$.
    \item
      The symmetric functions~$h_1, h_2, \dotsc$ generate~$\Lambda$ and are algebraically independent.
    \item
      The monomials~$h_\lambda$ where~$\lambda$ ranges through all partitions is a~{\basis{$\Integer$}} of~$\Lambda$.
    \item
      The symmetric functions~$s_\lambda$ where~$\lambda$ ranges through all partitions forms a~{\basis{$\Integer$}} of~$\Lambda$.
  \end{enumerate}
\end{proposition}

The mapping~$e_i(x_1, \dotsc, x_k) \mapsto e_i(x_1, \dotsc, x_{k+1})$ with~$i = 0, \dotsc, k$ gives an embedding of rings~$\Lambda(k) \to \Lambda(k+1)$, and the above shows that one can regard~$\Lambda$ is the resulting colimit, i.e.\ as the polynomial ring~$\Integer[e_1, e_2, e_3, \dotsc]$.
Similarly for~$h_i$ instead of~$e_i$.

\begin{theorem}
  Let~$\Phi \colon \Lambda \to R$ be the unique additive group homomorphism that maps the basis element~$e_\lambda$ to the element~$[\widetilde{M}]$.
  \begin{enumerate}
    \item
      The map~$\Phi$ is an isomorphism of rings.
    \item
      It holds that~$\Phi(h_\lambda) = [M^\lambda]$.
    \item
      It holds that~$\Phi(s_\lambda) = [S^\lambda]$.
  \end{enumerate}
\end{theorem}





\section{The goal}

\begin{corollary}
  The composition
  \[
    \Lambda
    \xlongto{\Phi^{-1}}
    R
    \xlongto{\Schur}
    \K_0(\GL(V))
    \xlongto{\Char}
    \Lambda(m)
  \]
  is given by~$f \mapsto f(x_1, \dotsc, x_m)$.
\end{corollary}

\begin{proof}
  The assertion holds for the basis elements~$s_\lambda$ of~$\Lambda$ as
  \[
    s_\lambda
    \mapsto
    [S^\lambda]
    \mapsto
    [\Schur(S^\lambda)]
    =
    [S^\lambda(V)]
    \mapsto
    s_\lambda(x_1, \dotsc, x_m) \,.
  \]
  The general assertion follow by additivity of all occuring maps.
\end{proof}

We have thus finally arrived at the following commutative diagram of rings:
\[
  \begin{tikzcd}[sep = huge]
    R
    \arrow[leftrightarrow]{r}{\sim}
    \arrow{d}[left]{\Schur}
    &
    \Lambda
    \arrow{d}[right]{f \mapsto f(x_1, \dotsc, x_m)}
    \\
    \K_0(\GL(V))
    \arrow{r}[above]{\sim}[below]{\Char}
    &
    \Lambda(m)
  \end{tikzcd}
\]
We have the following special cases of this diagram:
\begin{gather*}
  \begin{tikzcd}[ampersand replacement = \&]
    {[S^\lambda]}
    \arrow[leftrightarrow]{r}
    \arrow{d}
    \&
    s_\lambda
    \arrow{d}
    \\
    {[S^\lambda(V)]}
    \arrow[leftrightarrow]{r}
    \&
    s_\lambda(x_1, \dotsc, x_n)
  \end{tikzcd}
  \\
  \begin{tikzcd}[ampersand replacement = \&]
    {[\widetilde{M}^\lambda]}
    \arrow[leftrightarrow]{r}
    \arrow{d}
    \&
    e_\lambda
    \arrow{d}
    \\
    {[\widetilde{M}^\lambda(V)]}
    \arrow[leftrightarrow]{r}
    \&
    e_\lambda(x_1, \dotsc, x_n)
  \end{tikzcd}
  \qquad
  \begin{tikzcd}[ampersand replacement = \&]
    {[M^\lambda]}
    \arrow[leftrightarrow]{r}
    \arrow{d}
    \&
    h_\lambda
    \arrow{d}
    \\
    {[M^\lambda(V)]}
    \arrow[leftrightarrow]{r}
    \&
    h_\lambda(x_1, \dotsc, x_n)
  \end{tikzcd}
\end{gather*}

We can now use these correspondeces to translate between problems about the representation theory of~$\symm_n$, the representation theory of~$\GL(V)$, and the combinatorics of symmetric polynomials.

\begin{example}
  \leavevmode
  \begin{enumerate}
    \item
      For every~$n \geq 0$ there exists unique natural numbers~$f^\lambda$ for~$\lambda \ispar n$ such that one and thus all of the following conditions hold:
      \begin{enumerate}
        \item
          $\Complex[S_n] \cong \bigoplus_{\lambda \ispar n} (S^\lambda)^{\oplus f^\lambda}$,
        \item
          $V^{\tensor n} \cong \bigoplus_{\lambda \ispar n} S^\lambda(V)^{\oplus f^\lambda}$,
        \item
          $(x_1 + \dotsb + x_m)^n = \sum_{\lambda \ispar n} f^\lambda s_\lambda(x_1, \dotsc, x_m)$ for every~$m \geq 0$,
        \item
          $e_1^n = \sum_{\lambda \ispar n} f^\lambda s_\lambda$ in~$\Lambda$.
      \end{enumerate}
      We have already seen from the first description that~$f^\lambda$ is the number of standard Young tableaux of shape~$\lambda$.
    \item
      For every partition~$\lambda$ there exist unique natural numbers~$K_{\mu, \lambda}$ such that one and thus all of the following conditions hold:
      \begin{enumerate}
        \item
          $M^\lambda \cong S^\lambda \oplus \bigoplus_{\mu \strictlydominated \lambda} (S^\mu)^{\oplus K_{\mu, \lambda}}$,
        \item
          $M^\lambda(V) \cong S^\lambda(V) \oplus \bigoplus_{\mu \strictlydominated \lambda} S^\mu(V)^{\oplus K_{\mu, \lambda}}$,
        \item
          $h_\lambda(x_1, \dotsc, x_m) = s_\lambda(x_1, \dotsc, x_m) + \sum_{\mu \strictlydominated \lambda} K_{\mu, \lambda} s_\mu(x_1, \dotsc, x_m)$,
        \item
          $h_\lambda = s_\lambda + \sum_{\mu \strictlydominated \lambda} K_{\mu, \lambda} s_\mu$.
      \end{enumerate}
      The numbers~$K_{\mu, \lambda}$ are the \defemph{Kostka numbers}.
    \item
      For every partition~$\lambda$ there exist unique natural numbers~$\tilde{K}_{\mu, \lambda}$ such that one and thus all of the following conditions hold:
      \begin{enumerate}
        \item
          $\widetilde{M}^\lambda \cong S^\lambda \oplus \bigoplus_{\mu \strictlydominated \lambda} (S^\mu)^{\oplus \tilde{K}_{\mu, \lambda}}$,
        \item
          $\widetilde{M}^\lambda(V) \cong S^\lambda(V) \oplus \bigoplus_{\mu \strictlydominated \lambda} S^\mu(V)^{\oplus \tilde{K}_{\mu, \lambda}}$,
        \item
          $e_\lambda(x_1, \dotsc, x_m) = s_\lambda(x_1, \dotsc, x_m) + \sum_{\mu \strictlydominated \lambda} \tilde{K}_{\mu, \lambda} s_\mu(x_1, \dotsc, x_m)$,
        \item
          $e_\lambda = s_\lambda + \sum_{\mu \strictlydominated \lambda} \tilde{K}_{\mu, \lambda} s_\mu$.
      \end{enumerate}
      The numbers~$\tilde{K}_{\mu, \lambda}$ are again the Kostka numbers, connected to the above via~$\tilde{K}_{\mu, \lambda} = K_{\tilde{\mu}, \lambda}$ where~$\tilde{\mu}$ denotes the transposed of~$\mu$.
    \item
      For any two partitions~$\lambda$ ad~$\mu$ there exist natural numbers~$c^{\nu}_{\lambda, \mu}$ such that one and thus all of the following conditions holds:
      \begin{enumerate}
        \item
          $S^\lambda \circ S^\mu \cong \sum_\nu (S^\nu)^{\oplus c_{\lambda, \mu}^{\nu}}$,
        \item
          $S^\lambda(V) \tensor S^\mu(V) \cong \sum_\nu S^\nu(V)^{\oplus c_{\lambda, \mu}^{\nu}}$,
        \item
          $s_\lambda(x_1, \dotsc, x_m) s_\mu(x_1, \dotsc, x_m) = \sum_{\nu} c_{\lambda, \mu}^\nu s_\nu(x_1, \dotsc, x_m)$,
        \item
          $s_\lambda s_\mu = \sum_{\nu} c_{\lambda, \mu} s_\nu$.
      \end{enumerate}
      The numbers~$c_{\lambda, \mu}^{\nu}$ are the \defemph{Littlewood--Richardson coefficients}.
  \end{enumerate}
\end{example}

\end{document}
