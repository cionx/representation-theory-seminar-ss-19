\documentclass[a4paper,10pt]{scrartcl}

\usepackage{style}

\titlehead{Graduate Seminar on Representation Theory \hfill Jendrik Stelzner \hfill July 4, 2019}
\title{More on the \\ Representation~Theory \\ of~$\GL(V)$}
\author{}
\date{}

\begin{document}

\maketitle

\vspace{-4em}

We’re working over the fixed ground field~$\Complex$.
We abbreviate~$\tensor_{\Complex}$ as~$\tensor$.
We denote by~$V$ some~{\dimensional{$m$}} vector space.
The set of natural numbers~$\Natural$ contains~$0$.
We write~$\lambda \ispar n$ to mean that~$\lambda$ is a partition of~$n$.

In this talk we will continue to explain the following connections:
\[
  \begin{tikzcd}[sep = large]
    \begin{tabular}{c}
      representation theory of \\
      the symmetric groups
    \end{tabular}
    \arrow[leftrightarrow]{r}
    \arrow{d}
    &
    \text{symmetric functions}
    \arrow{d}
    \\
    \begin{tabular}{c}
      representation theory \\
      of~$\GL(V)$
    \end{tabular}
    \arrow[leftrightarrow]{r}
    &
    \text{symmetric polynomials}
  \end{tikzcd}
\]





\section{Recalling the last talk}



\subsection{The irreducible representations~$S^\lambda(V)$}

We have previously seen that the isomorphism classes of irreducible finite-dimensional polynomial representations of~$\GL(V)$ are indexed by the set of all partitions:
For every partition~$\lambda$ with~$\lambda = (\lambda_1, \dotsc, \lambda_t)$ let
\[
  \widetilde{M}^\lambda(V)
  =
  \Exterior^{\tilde{\lambda}_1}(V) \tensor \dotsb \tensor \Exterior^{\tilde{\lambda}_n}(V) \,.
\]
where~$\tilde{\lambda}$ is the transposed of~$\lambda$.
The tensor factor~$\Exterior^{\tilde{\lambda}_j}(V)$ hence comes from the~{\howmanyth{$j$}} column of the Young diagram associated to~$\lambda$, which has height~$\tilde{\lambda}_j$.
Note that~$\widetilde{M}^\lambda(V) = 0$ if~$\length(\lambda) > m$, i.e.\ if the Young diagram of~$\lambda$ has more than~$m$ rows.

\begin{example}
  The Young diagram of~$\lambda = (3,3,2,1)$ is given by
  \[
    \ydiagram{3,3,2,1}
  \]
  and therefore
  \[
    \ytableausetup{boxsize=0.7em}
    \widetilde{M}^{\lambda}(V)
    =
    \Exterior^4(V) \tensor \Exterior^3(V) \tensor \Exterior^2(V) \,.
    \ytableausetup{boxsize=normal}
  \]
\end{example}

The \defemph{Schur module} associated to~$\lambda$ is given by
\[
  S^{\lambda}(V)
  =
  \widetilde{M}^{\lambda}(V)/Q^{\lambda}(V)
\]
for a certain submodule~$Q_\lambda$ of~$\widetilde{M}_\lambda$.
For~$\length(\lambda) \leq m$ these are precisely the irreducible finite-dimensional polynomial representations of~$\GL(V)$.

\begin{example}
  \leavevmode
  \begin{enumerate}
    \item
      For~$\lambda = (1, \dotsc, 1) \ispar n$ we have~$S^{\lambda}(V) = \Exterior^n(V)$.
    \item
      For~$\lambda = (n)$ we have~$S^{\lambda}(V) = \Symm^n(V)$.
  \end{enumerate}
\end{example}



\subsection{Characters}

The polynomial~{\representations{$\GL(V)$}} can also be described by their characters:
Using a basis~$e_1, \dotsc, e_m$ of~$V$ we may identify~$\GL(V)$ with~$\GL_m(\Complex)$.
For any finite-dimensional polynomial representation~$\rho \colon \GL(V) \to \GL(W)$ the \defemph{weight space}~$W_\beta$ for~$\beta = (\beta_1, \dotsc, \beta_m) \in \Natural^m$ is given by
\[
  W_\beta
  =
  \{
    w \in W
  \suchthat
    \text{$\diag(x_1, \dotsc, x_m).w = x_1^{\beta_1} \dotsm x_m^{\beta_m} w$ for all~$x_1, \dotsc, x_m \in \Complex^{\times}$}
  \} \,.
\]
The representation~$W$ decomposes into weight spaces in the sense that~$W = \bigoplus_{\beta \in \Natural^m} W_\beta$ and its \defemph{character} is the polynomial~$\Char(W) \in \Integer[x_1, \dotsc, x_m]$ given by
\[
  \Char(W)(x_1, \dotsc, x_m)
  =
  \tr(\rho(\diag(x_1, \dotsc, x_m)))
  =
  \sum_{\beta \in \Natural^m} \dim(W_\beta) x^\beta \,.
\]
The character~$\Char(W)$ is actually a symmetric polynomial, and the representation~$W$ is (up to isomorphism) uniquely determined by its character.

\begin{example}
  \leavevmode
  \begin{enumerate}
    \item
      The basis~$e_{i_1} \wedge \dotsb \wedge e_{i_n}$ of~$\Exterior^n(V)$ with~$i_1 < \dotsb < i_n$ consists of weight vectors, and its character is given by
      \[
        \Char\Bigl( \Exterior^n(V) \Bigr)(x_1, \dotsc, x_m)
        =
        \sum_{1 \leq i_1 < \dotsb < i_n \leq m} x_{i_1} \dotsb x_{i_n}
        =
        e_n(x_1, \dotsc, x_m) \,,
      \]
      the~{\howmanyth{$n$}} elementary symmetric polynomial in~$m$ variables.
    \item
      We see similarly that
      \[
        \Char(\Symm^n(V))(x_1, \dotsc, x_m)
        =
        \sum_{1 \leq i_1 \leq \dotsb \leq i_n \leq m} x_{i_1} \dotsb x_{i_n}
        =
        h_n(x_1, \dotsc, x_m)
      \]
      is the~{\howmanyth{$n$}} complete homogeneous symmetric polynomial in~$m$ variables.
  \end{enumerate}
\end{example}





\section{From~$\symm_n$ to~$\GL(V)$}

We denote for any group~$G$ by~$\trivrep_G$ the trivial representation of~$G$.
We denote for~$n \geq 1$ by~$\signrep_{\symm_n}$ the sign representation of the symmetric group~$\symm_n$.

For every~$n \geq 0$ the tensor power~$V^{\tensor n}$ is again a~{\representation{$\GL(V)$}}, and it is also a right~{\representation{$\symm_n$}} via
\[
  (e_1 \tensor \dotsb \tensor e_n) . \sigma
  =
  e_{\sigma(1)} \tensor \dotsb \tensor e_{\sigma(n)} \,.
\]
It follows that for every~{\representation{$\symm_n$}}~$E$ we get a~{\representation{$\GL(V)$}}
\[
  \Schur(E)
  \defined
  V^{\tensor n} \tensor_{\Complex[\symm_n]} E \,.
\]
This construction results an exact additive functor
\[
  \Schur
  \colon
  \rep{\symm_n}
  \to
  \prep{\GL(V)} \,,
\]
where~$\rep{\symm_n}$ is the category of finite-dimensional~{\representations{$\symm_n$}} and~$\rep{\GL(V)}$ denotes the category of finite-dimensional polynomial~{\representations{$\GL(V)$}}. 
The functor~$\Schur$ is the \defemph{Schur functor}.
It depends on both~$n$ and~$V$, but we will omit this in our notation.

\begin{example}
  \leavevmode
  \begin{enumerate}
    \item
      We have that~$\trivrep_{\symm_n} \cong \Complex[\symm_n]/\gen{\sigma - 1 \suchthat \sigma \in \symm_n}_{\Complex}$ and thus
      \begin{align*}
        \Schur( \trivrep_{\symm_n} )
        &\cong
        V^{\tensor n} \tensor_{\Complex[\symm_n]} \Complex[\symm_n]/\gen{\sigma - 1 \suchthat \sigma \in \symm_n}_{\Complex}
        \\
        &\cong
        V^{\tensor n}/\gen{x - x.\sigma \suchthat t \in V^{\tensor n}}_{\Complex}
        \\
        &\cong
        \Symm^n(V) \,.
      \end{align*}
    \item
      We find similarly that~$\Schur( \signrep_{\symm_n} ) \cong \Exterior^n(V)$.
  \end{enumerate}
\end{example}

If~$V_1, \dotsc, V_t$ are representations of the groups~$\symm_{n_1}, \dotsc, \symm_{n_t}$ then for~$n = n_1 + \dotsb + n_t$,
\[
  V_1 \circ \dotsb \circ V_t
  \defined
  \Complex[\symm_n]
  \tensor_{\Complex[\symm_{n_1} \times \dotsb \times \symm_{n_t}]}
  (V_1 \outertensor \dotsb \outertensor V_t)
\]
is a representation of~$\symm_n$.

\begin{lemma}
  \label{properties of circle product}
  Let~$E_i$,~$E$ and~$F$ be representations of symmetric groups.
  Then
  \begin{enumerate}
    \item
      $(E_1 \circ \dotsb \circ E_s) \circ \dotsb \circ (F_1 \circ \dotsb \circ F_t) \cong E_1 \circ \dotsb \circ F_t$,
    \item
      $E \circ F \cong F \circ E$.
  \end{enumerate}
\end{lemma}

\begin{lemma}
  \label{multiplicativity of schur functor}
  If~$E$ and~$F$ are representations of symmetric groups~$\symm_a$ and~$\symm_b$ then
  \[
    \Schur(E \circ F) \cong \Schur(E) \tensor \Schur(F) \,.
  \]
\end{lemma}

\begin{proof}
  We find that
  \begin{align*}
    \Schur(E \circ F)
    &=
    V^{\tensor (a+b)}
    \tensor_{\Complex[\symm_{a+b}]}
    \Complex[\symm_{a+b}]
    \tensor_{\Complex[\symm_a \times \symm_b]}
    (E \outertensor F)
    \\
    &\cong
    V^{\tensor (a+b)}
    \tensor_{\Complex[\symm_a \times \symm_b]}
    (E \outertensor F)
    \\
    &\cong
    (V^{\tensor a} \outertensor V^{\tensor b})
    \tensor_{\Complex[\symm_a] \tensor \Complex[\symm_b]}
    (E \outertensor F)
    \\
    &\cong
    (V^{\tensor a} \tensor_{\Complex[\symm_a]} E)
    \tensor
    (V^{\tensor b} \tensor_{\Complex[\symm_b]} F)
    \\
    &=
    \Schur(E) \tensor \Schur(F)
  \end{align*}
  as claimed.
\end{proof}





\section{Some representation theory of~$\symm_n$}

For better use of the Schur functor~$\Schur$ we need some understand of the representation theory of the symmetric group~$\symm_n$, where~$n \geq 0$.
In this section we always assume that~$\lambda \ispar n$ unless otherwise specified.

Recall that a \defemph{semistandard Young tableaux} of shape~$\lambda$ assigns to each box of the Young diagram of~$\lambda$ a natural number~$1, 2, 3, \dotsc$, weakly increasing in each row and strictly increasing in each column.
A \defemph{numbering} of shape~$\lambda$ fills in the boxes of the Young diagram of~$\lambda$ bijectively with the numbers~$1, \dotsc, n$.
A \defemph{standard Young tableaux} is a semistandard Young tableaux that is also a numbering.
This means that we assign to the boxes of the Young diagram of~$\lambda$ bijectively the integers~$1, 2, \dotsc, n$, such that both rows and columns are (necessarily strictly) increasing.
\[
  \begin{tikzcd}[column sep = small]
    \text{numbering}
    \arrow[dash]{dr}
    &
    {}
    &
    \begin{tabular}{@{}c@{}}
      semistandard \\
      Young tableaux
    \end{tabular}
    \arrow[dash]{dl}
    \\
    {}
    &
    \begin{tabular}{@{}c@{}}
      standard \\
      Young tableaux
    \end{tabular}
    &
    {}
  \end{tikzcd}
\]


\subsection{The representation $M^\lambda$}

Two numbering~$T$ and~$T'$ of a Young diagram of shape~$\lambda$ are \defemph{row-equivalent} if they have the same entries in each row.
A \defemph{row tabloid} is an equivalence class of row-equivalent numbering.
A row tabloid can be represented as follows:
\[
  \Tabloid{{1,2,3},{4,5,6},{7,8}}
  =
  \Tabloid{{2,1,3},{4,6,5},{8,7}}
\]
Every numbering~$T$ defines a row tabloid~$[T]$.

The group~$\symm_n$ acts transitive on the set of numberings of shape~$\lambda$, which induces a transitive group action on the set of row tabloids of shape~$\lambda$.
The \defemph{row group}~$R(T)$ of a numbering~$T$ is given by all permutations~$\sigma \in \symm_n$ that act row-wise on~$T$, i.e.\ the stabilizer of the associated tabloid~$[T]$.

It follows that~$\symm_n$ acts linearly on~$M^\lambda$, the free vector space on the set of row tabloids of shape~$\lambda$, and that for any numbering~$T$ of shape~$\lambda$,
\begin{align*}
  M^\lambda
  &\cong
  \Complex[\symm_n]/(p - 1 \suchthat p \in R(T))
  \\
  &\cong
  \Complex[\symm_n] \tensor_{\Complex[R(T)]} \Complex[R(T)]/(p - 1 \suchthat p \in R(T))
  \\
  &\cong
  \Complex[\symm_n] \tensor_{\Complex[R(T)]} \trivrep_{R(T)} \,.
\end{align*}
If~$T$ is the \enquote{horizontal standard numbering}, e.g.
\[
  \begin{ytableau}
     1 &  2 & 3 & 4 & 5 \\
     6 &  7 & 8 & 9 \\
  \end{ytableau}
\]
for~$\lambda = (5,4)$, then~$R(T) = \symm_{\lambda_1} \times \dotsb \times \symm_{\lambda_t}$ where~$\lambda = (\lambda_1, \dotsc, \lambda_t)$ and thus
\begin{align*}
  M^\lambda
  &\cong
  \Complex[\symm_n] \tensor_{\Complex[R(T)]} \trivrep_{R(T)}
  \\
  &\cong
  \Complex[\symm_n]
  \tensor_{\Complex[\symm_{\lambda_1} \times \dotsb \times \symm_{\lambda_t}]}
  \trivrep_{\symm_{\lambda_1} \times \dotsb \times \symm_{\lambda_t}}
  \\
  &\cong
  \Complex[\symm_n]
  \tensor_{\Complex[\symm_{\lambda_1} \times \dotsb \times \symm_{\lambda_t}]}
  (\trivrep_{\symm_{\lambda_1}} \outertensor \dotsb \outertensor \trivrep_{\symm_{\lambda_t}})
  \\
  &=
  \trivrep_{\symm_{\lambda_1}} \circ \dotsb \circ \trivrep_{\symm_{\lambda_t}} \,.
\end{align*}
It follows that
\begin{align*}
  \Schur(M^\lambda)
  &\cong
  \Schur(\trivrep_{\symm_{\lambda_1}} \circ \dotsb \circ \trivrep_{\symm_{\lambda_t}})
  \\
  &\cong
  \Schur(\trivrep_{\symm_{\lambda_1}}) \tensor \dotsb \tensor \Schur(\trivrep_{\symm_{\lambda_t}})
  \\
  &\cong
  \Symm^{\lambda_1}(V) \tensor \dotsb \tensor \Symm^{\lambda_t}(V)
  \\
  &\defines
  M^\lambda(V) \,,
\end{align*}
or alternatively that
\begin{align*}
  \Schur(M^\lambda)
  &=
  V^{\tensor n} \tensor_{\Complex[\symm_n]} \Complex[\symm_n]/(p - 1 \suchthat p \in R(T))
  \\
  &\cong
  V^{\tensor n}/(x - x.p \suchthat p \in R(T))
  \\
  &\cong
  V^{\tensor n}/(x - x.p \suchthat p \in \symm_{\lambda_1} \times \dotsb \times \symm_{\lambda_t})
  \\
  &\cong
  \Symm^{\lambda_1}(V) \otimes \dotsb \otimes \Symm^{\lambda_t}(V) \,.
\end{align*}



\subsection{The representation~$\widetilde{M}^\lambda$}

We can alter the construction of~$M^\lambda$ in two ways:
Working with column instead of rows and introducing an alternating sign.

We denote by~$C(T)$ the column group of a numbering~$T$, i.e.\ all permutations that act column-wise on~$T$.
We let~$\widetilde{M}^\lambda$ be the free vector generated by the numbering~$T$ of shape~$\lambda$ subject to the relations~$\sigma.T = \sign(\sigma) T$ for~$\sigma \in C(T)$.
The resulting vector space generators~$[T]$ of~$\widetilde{M}^\lambda$ may be visualized as follows:
\[
  \ColumnTabloid{{1,4},{2,5},{3}}
  =
  - \ColumnTabloid{{2,4},{1,5},{3}}
  =
  \ColumnTabloid{{2,5},{1,4},{3}}
\]
The group~$\symm_n$ acts on~$\widetilde{M}^\lambda$ via~$\sigma.[T] = [\sigma.T]$ for any numbering~$T$ of shape~$\lambda$, and
\begin{align*}
  \widetilde{M}^\lambda
  &\cong
  \Complex[\symm_n]/(q - \sign(q)1 \suchthat q \in C(T))
  \\
  &\cong
  \signrep_{\lambda_1} \circ \dotsb \circ \signrep_{\lambda_t} \,.
\end{align*}
It follows that
\[
  \Schur\bigl( \widetilde{M}^\lambda \bigr)
  \cong
  \Exterior^{\lambda_1}(V) \tensor \dotsb \tensor \Exterior^{\lambda_t}(V)
  =
  \widetilde{M}^{\lambda}(V) \,.
\]



\subsection{Specht modules}

The irreducible representations of the symmetric group~$\symm_n$ can be indexed by the partitions~$\lambda \ispar n$ and constructed as follows:

If~$T$ is any numbering of shape~$\lambda$ then its \defemph{Young centralizer} is the element
\[
  c_T
  \defined
  \sum_{q \in C(T)} \sign(q) q
  \in
  \Complex[\symm_n] \,.
\]
The \defemph{Specht module}~$S^{\lambda}$ is the subspace of~$M^{\lambda}$ spanned by the elements~$v_T \defined c_T \cdot [T]$ as~$T$ ranges through the numbering of shape~$\lambda$.
It holds that~$\sigma \cdot v_T = v_{\sigma \cdot T}$ for every~$\sigma \in \symm_n$ and numbering~$T$ of shape~$\lambda$ whence~$S^\lambda$ is a subrepresentation of~$M^\lambda$.

\begin{theorem}[Classification of irreducible representations of~$\symm_n$]
  \label{irreps of sn}
  \leavevmode
  \begin{enumerate}
    \item
      The representations~$S^\lambda$ with~$\lambda \ispar n$ are pairwise non-isomorphic irreducible representations of the symmetric group~$\symm_n$, and every irreducible representation of~$\symm_n$ is of this form.
    \item
      The elements~$v_T$ where~$T$ ranges through the standard Young tableaux of shape~$\lambda$ form a basis of~$S^\lambda$.
      Hence
      \[
        \dim S^\lambda
        =
        \text{number of standard Young tableaux of shape~$\lambda$} \,.
      \]
  \end{enumerate}
\end{theorem}

One can also construct the Specht module~$S^{\lambda}$ as a quotient of the representation~$\widetilde{M}^\lambda$:
The map
\[
  \widetilde{M}^\lambda
  \to
  S^\lambda
  \quad
  [T]
  \mapsto
  v_T
\]
is a well-defined surjective homomorphism of~{\representations{$\symm_n$}}.
Its kernel~$Q^\lambda$ can be described similarly to~$Q^\lambda(V)$, and it follows that under the isomorphism~$\Schur(\widetilde{M}^\lambda) \cong \widetilde{M}^\lambda(V)$ we have~$\Schur(Q^\lambda) \cong Q^{\lambda}(V)$.

\begin{theorem}
  For any partition~$\lambda \ispar n$,
  \[
    \Schur(S^\lambda) \cong S^{\lambda}(V) \,.
  \]
\end{theorem}

\begin{proof}
  By applying the (exact) functor~$\Schur(-)$ to the short exact sequence
  \[
    0
    \to
    Q^\lambda
    \to
    \widetilde{M}^\lambda
    \to
    S^\lambda
    \to
    0
  \]
  we get the short exact sequence
  \[
    0
    \to
    \Schur(Q^\lambda)
    \to
    \Schur(\widetilde{M}^\lambda)
    \to
    \Schur(S^\lambda)
    \to
    0
  \]
  and find that
  \[
    \Schur(S^\lambda)
    \cong
    \Schur(\widetilde{M}^\lambda)/\Schur(Q^\lambda)
    \cong
    \widetilde{M}^{\lambda}(V)/Q^{\lambda}(V)
    \cong
    S^{\lambda}(V) \,,
  \]
  proving the assertion.
\end{proof}


\begin{remark}
  \label{kernel of schur functor}
  If~$\length(\lambda) > m$, i.e.\ if the Young diagram of~$\lambda$ has more than~$m = \dim(V)$ rows, then~$\Schur(S^\lambda) = S^\lambda(V) = 0$.
  If~$\length(\lambda) \leq m$ then~$\Schur(S^\lambda) = S^\lambda(V)$ is again an irreducible representation.
\end{remark}


\begin{corollary}
  For any~$n \geq 0$,~$V^{\tensor n} \cong \bigoplus_{\lambda \ispar n} S^{\lambda}(V)^{\oplus f_\lambda}$ where~$f_\lambda$ denotes the number of standard Young tableaux of shape~$\lambda$.
\end{corollary}


\begin{proof}
  We find with \cref{irreps of sn} and Maschke’s theorem that
  \[
    \Complex[\symm_n]
    \cong
    \bigoplus_{\lambda \ispar n} (S^\lambda)^{\oplus f^\lambda}
  \]
  with
  \begin{align*}
    f^\lambda
    &=
    \text{multiplicity of~$S_\lambda$ in~$\Complex[\symm_n]$}
    \\
    &=
    \text{dimension of~$S_\lambda$}
    \\
    &=
    \text{number of standard Young tableaux of shape~$\lambda$}
  \end{align*}
  where the second equality follows (for example) from the Artin--Wedderburn theorem.
  It follows that
  \begin{align*}
    V^{\tensor n}
    &=
    V^{\tensor n} \tensor_{\Complex[\symm_n]} \Complex[\symm_n]
    \\
    &\cong
    \Schur(\Complex[\symm_n])
    \\
    &\cong
    \Schur\left( \bigoplus_{\lambda \ispar n} (S^\lambda)^{\oplus f^\lambda} \right)
    \\
    &\cong
    \bigoplus_{\lambda \ispar n} \Schur(S^\lambda)^{\oplus f^\lambda}
    \\
    &\cong
    \bigoplus_{\lambda \ispar n} S^{\lambda}(V)^{\oplus f^\lambda}
  \end{align*}
  as claimed.
\end{proof}


\begin{example}
  The partitions of~$n = 2$ are~$\lambda = (1,1)$ and~$\mu = (2)$.
  Each of those partitions admits precisely one standard Young tableau:
  \[
    \begin{ytableau}
      1 & 2
    \end{ytableau}
    \qquad
    \begin{ytableau}
      1 \\
      2
    \end{ytableau}
  \]
  It follows that~$f_\lambda = f_\mu = 1$ and therefore
  \[
    V \tensor V
    =
    V^{\tensor 2}
    \cong
    S^{\lambda}(V)^{\oplus f_\lambda}
    \oplus
    S^{\mu}(V)^{\oplus f_\mu}
    \cong
    \Symm^2(V) \oplus \Exterior^2(V) \,.
  \]
\end{example}


\begin{remark}
  The Schur functor~$\Schur = V^{\tensor n} \tensor_{\Complex[S_n]} (-)$ admits a right adjoint given by~$\Schur' = \Hom_{\GL(V)}(V^{\tensor n}, -)$.
  We have for every partition~$\lambda$ with~$\lambda \ispar n$ and~$\ell(\lambda) \leq m$ and every finite-dimensional polynomials~{\representation{$\GL(V)$}}~$W$ that
  \begin{align*}
    \text{multiplicity of~$S^\lambda$ in~$\Schur'(W)$}
    &=
    \dim \Hom_{\symm_n}(S^\lambda, \Schur'(W))
    \\
    &=
    \dim \Hom_{\GL(V)}(\Schur(S^\lambda), W)
    \\
    &=
    \dim \Hom_{\GL(V)}(S^\lambda(V), W)
    \\
    &=
    \text{multiplicity of~$S^\lambda(V)$ in~$W$}
  \end{align*}
  We see in particular that~$\Schur'(S^\lambda(V)) \cong S^\lambda$.
\end{remark}





\section{Translation into rings}



\subsection{The Grothendieck ring of~$\GL(V)$}

Recall that the Grothendieck group~$\K_0(\Acat)$ of an abelian category~$\Acat$ is generated by the isomorphism classes of objects of~$\Acat$ subject to the relation~$[B] = [A] + [C]$ for every short exact sequence~$0 \to A \to B \to C \to 0$ in~$\Acat$.
Then in particular
\begin{equation}
  \label{additivity in grothendieck group}
  [A] + [B] = [A \oplus B]
\end{equation}
for all~$A, B \in \Acat$.
If the abelian category~$\Acat$ is semisimple, i.e.\ if every short exact sequence in~$\Acat$ splits, then the relations~\eqref{additivity in grothendieck group} sufficies to define~$\K_0(\Acat)$.

Let~$\Acat$ be the semisimple category of finite-dimensional polynomial representations of~$\GL(V)$.
We abbreviate~$\K_0(\GL(V)) \defined \K_0(\Acat)$
This becomes a ring when endowed with the multiplication
\[
  [V] \cdot [W]
  =
  [V \tensor W] \,.
\]
It follows from last week’s classification of irreducible finite-dimensional polynomial~{\representations{$\GL(V)$}} that~$\K_0(\GL(V))$ admits a~{\basis{$\Integer$}} given by the isomorphism classes~$[S^\lambda(V)]$ where~$\lambda$ ranges through all partitions with~$\length(\lambda) \leq m$.

The character of a~{\representation{$\GL(V)$}} results in a well-defined ring homomorphism
\[
  \Char
  \colon
  \K_0(\GL(V))
  \to
  \Lambda(m)
\]
where~$\Lambda(m)$ denotes the ring of symmetric polynomials in the variables~$x_1, \dotsc, x_m$.


\subsection{The representation ring of~$\symm_n$}

For every~$n \geq 0$ let~$R_n$ be the Grothendieck group of~$\rep{S_n}$, the category of finite-dimensional~{\representations{$S_n$}}.
We can define on~$R \defined \bigoplus_{n \geq 0} R_n$ a multiplication via
\[
  [E] \cdot [F]
  =
  [E \circ F] \,.
\]
Note that~$1_R = [\trivrep_{S_0}]$ is multiplicative neutral and that this multiplicaton is associative and commutative by \cref{properties of circle product}.
The multiplication is also distributive and~$R_i R_j \subseteq R_{i+j}$ for all~$i,j \geq 0$.
Hence~$R$ becomes a graded ring.

The category~$\rep{S_n}$ is semisimple by Maschke’s theorem and it follows from \cref{irreps of sn} that~$R_n$ admits a~{\basis{$\Integer$}} given by the isomorphism classes~$[S^\lambda]$ with~$\lambda \ispar n$.
The ring~$R$ hence damits a~{\basis{$\Integer$}} given by the isomorphism classes~$[S^\lambda]$ where~$\lambda$ ranges through all partitions (of all natural numbers).

The additivity of the Schur functor(s)~$\Schur \colon \rep{S_n} \to \Acat$ gives for every~$n \geq 0$ a group homomorphism~$R_n \to \K_0(\GL(V))$, which together give a group homomorphism~$R \to \K_0(\GL(V))$.
It follows from \cref{multiplicativity of schur functor} that this is a ring homomorphism.
By abuse of notation we denote this homomorphism again by~$\Schur$.

We have argued in \cref{kernel of schur functor} that the kernel of~$\Schur \colon R \to \K_0(\GL(V))$ is spanned by those~$[S^\lambda]$ with~$\length(\lambda) > m$, whereas all other basis vector~$[S^\lambda]$ with~$\length(\lambda) \leq m$ are mapped bijectively onto the basis~$[S^\lambda(V)]$ of~$\K_0(\GL(V))$.


\subsection{The ring of symmetric functions}

For every~$k \geq 0$ let~$\Lambda(k)$ denote the ring of symmetric polynomials, a subring of~$\Integer[x_1, \dotsc, x_k]$.

When dealing with symmetric polynomials it often happens that the number of variables,~$k$, does not matter:
One has a family~$(f_k)_{k \geq 0}$ of symmetric polynomials~$f_k \in \Lambda(k)$ such that~$f_k(x_1, \dotsc, x_{k-1}, 0) = f_{k-1}(x_1, \dotsc, x_{k-1})$ for every~$k \geq 1$, and for every~$k \geq 1$ an identity involving~$f_k$ which reduces for~$x_k \to 0$ to the identity involving~$f_{k-1}$.

\begin{example}
  The elementary symmetric polynomials~$e_n(x_1, \dotsc, x_k)$ and the completely symmetric polynomials~$h_n(x_1, \dotsc, x_k)$ are for every~$k \geq 1$ related by the formulas
  \[
    \sum_{i=0}^s (-1)^i e_i(x_1, \dotsc, x_k) h_{s-i}(x_1, \dotsc, x_k) = 0
  \]
  where~$s \geq 0$.
  The identities in~$k-1$ variables follows from the corresponding ones in~$k$ variables by setting~$x_k \to 0$.
\end{example}

To formalize this phenomenon we introduce the \defemph{ring of symmetric functions}~$\Lambda$:
For every degree~$n \geq 0$ we set
\begin{align*}
  \Lambda_n
  &\defined
  \left\{
    (f_k)_{k \geq 0}
  \suchthat*
    \begin{tabular}{c}
      $f_k \in \Lambda_n(k)$ with \\
      $f_k(x_1, \dotsc, x_{k-1}, 0) = f_{k-1}(x_1, \dotsc, x_{k-1})$ \\
      for every~$k \geq 1$
    \end{tabular}
  \right\}
  \\
  &=
  \lim( \Lambda_n(0) \from \Lambda_n(1) \from \Lambda_n(2) \from \Lambda_n(3) \from \dotsb)
\end{align*}
where~$\Lambda_n(k) \to \Lambda_n(k-1)$ is the group homomorphism given by setting  ~$x_k \to 0$.
We combine these groups into a graded ring~$\Lambda = \bigoplus_{n \geq 0} \Lambda_n$ with multiplication given by
\[
  (f_k)_{k \geq 0} \cdot (g_k)_{k \geq 0}
  \defined
  (f_k g_k)_{k \geq 0} \,.
\]13

\begin{remark}
  \leavevmode
  \begin{enumerate}
    \item
      We have
      \[
        \Lambda
        =
        \lim( \Lambda(0) \from \Lambda(1) \from \Lambda(2) \from \Lambda(3) \from \dotsb)
      \]
      in the category of graded rings.
    \item
      The elements of~$\Lambda$, the ring of symmetric functions, are not functions, despite the name.
  \end{enumerate}
\end{remark}

\begin{example}
  For every~$n \geq 0$ we have an element
  \[
    e_n
    \defined
    (e_n(), e_n(x_1), e_n(x_1, x_2), e_n(x_1, x_2, x_3), \dotsc)
    \in
    \Lambda_n
  \]
  and similarly elements~$h_n, p_n \in \Lambda_n$.
  We get for every partition~$\lambda = (\lambda_1, \dotsc, \lambda_t)$ an induced element
  \[
    e_{\lambda}
    \defined
    e_{\lambda_1} \dotsm e_{\lambda_t}
    \in
    \Lambda_{\size{\lambda}}
  \]
  and similarly elements~$h_{\lambda}, p_{\lambda} \in \Lambda_{\size{\lambda}}$.
\end{example}


\begin{example}[Schur polynomials  and Schur functions]
  Let~$\lambda$ be a partition.
  For every semistandard Young tableaux~$T$ of shape~$\lambda$ with entries in~$\{1, \dotsc, k\}$ let
  \[
    x_T
    \defined
    \prod_{i \in T} x_i \,.
  \]
  The \defemph{Schur polynomial}~$s_\lambda(x_1, \dotsc, x_k)$ is defined as
  \[
    s_\lambda(x_1, \dotsc, x_k)
    =
    \sum_{T'} x_{T'}
    \in
    \Integer[x_1, \dotsc, x_n]
  \]
  where~$T'$ ranges through the semistandard Young tableaux of shape~$\lambda$ with entries in~$\{1, \dotsc, k\}$.
  
  If for example~$\lambda = (2,2)$ and~$k = 3$ then the semistandard Young tableaux are as follows:
  \[
    \begin{ytableau}
      1 & 1 \\
      2 & 2
    \end{ytableau}
    \qquad
    \begin{ytableau}
      1 & 1 \\
      2 & 3
    \end{ytableau}
    \qquad
    \begin{ytableau}
      1 & 1 \\
      3 & 3
    \end{ytableau}
    \qquad
    \begin{ytableau}
      1 & 2 \\
      2 & 3
    \end{ytableau}
    \qquad
    \begin{ytableau}
      1 & 2 \\
      3 & 3
    \end{ytableau}
    \qquad
    \begin{ytableau}
      2 & 2 \\
      3 & 3
    \end{ytableau}
  \]
  The Schur polynomial~$s_{(2,2)}(x_1, x_2, x_3)$ is therefore given by
  \[
    s_{(2,2)}(x_1, x_2, x_3)
    =
    x_1^2 x_2^2 + x_1^2 x_2 x_3 + x_1^2 x_3^2 + x_1 x_2^2 x_3 + x_1 x_2 x_3^2 + x_2^2 x_3^3 \,.
  \]
  We observe the following:
  \begin{enumerate}
    \item
      The Schur polynomial~$s_\lambda(x_1, \dotsc, x_k)$ is homogeneous of degree~$\size{\lambda}$.
    \item
      If~$\length(\lambda) > k$, i.e.\ if the Young diagram of~$\lambda$ has more than~$k$ rows, then the Schur polynomial~$s_\lambda(x_1, \dotsc, x_k)$ vanishes since there exist no semistandard Young tableaux of shape~$\lambda$ with entries in~$\{1, \dotsc, k\}$.
      (We don’t have enough entries to make the first column strictly increasing, which is required for a semistandard Young tableaux.)
    \item
      The Specht~module~$S^\lambda(V)$ has a basis~$e_T$ where~$T$ ranges through the semistandard Young tableaux with entries in~$\{1, \dotsc, m\}$.
      Each~$e_T$ is a weight vector with corresponding weight~$x_T$.
      Hence
      \[
        s_\lambda(x_1, \dotsc, x_m)
        =
        \Char(S^\lambda(V))(x_1, \dotsc, x_m) \,.
      \]
      This shows in particular that~$s_\lambda(x_1, \dotsc, x_m)$ is a symmetric polynomial.
      We also see again that~$s_\lambda(x_1, \dotsc, x_m) = 0$ if~$\length(\lambda) > m$ since then~$S^\lambda(V) = 0$.
    \item
      It holds that~$s_\lambda(x_1, \dotsc, x_{k-1}, 0) = s_\lambda(x_1, \dotsc, x_{k-1})$.
  \end{enumerate}
  We find that we get a well-defined element~$s_\lambda \in \Lambda_{\size{\lambda}}$, the \defemph{Schur function} associated to~$\lambda$.
\end{example}

\begin{proposition}
  For every~$k \geq 0$ the ring of symmetric polynomials~$\Lambda(k)$ has the Schur polynomials~$s_\lambda(x_1, \dotsc, x_k)$ with~$\length(\lambda) \leq k$ as a basis.
\end{proposition}

\begin{corollary}
  The ring homomorphism~$\Char \colon \K_0(\GL(V)) \to \Lambda(m)$ is an isomorphim.
\end{corollary}

\begin{proof}
  The basis~$[S^\lambda(V)]$ of~$\K_0(\GL(V))$ is mapped to the basis~$s_\lambda(x_1, \dotsc, x_m)$ of~$\Lambda(m)$, both indexed by the partitions~$\lambda$ with~$\length(\lambda) \leq m$.
\end{proof}

We have for every~$k \geq 0$ a homomorphism of graded rings
\[
  \Lambda
  \to
  \Lambda(k) \,,
  \quad
  f
  \mapsto
  f(x_1, \dotsc, x_k)
\]
that assigns to~$f$ the entailed symmetric polynomial in~$k$ variables.
An equality~$f = g$ holds in~$\Lambda$ if and only if for every~$k \geq 0$ the equality~$f(x_1, \dotsc, x_k) = g(x_1, \dotsc, x_k)$ hold.

\begin{example}
  The identities~$\sum_{i=0}^s (-1)^i e_i h_{s-i} = 0$ for~$s \geq 0$ hold in~$\Lambda$.
\end{example}

\begin{proposition}
  \leavevmode
  \begin{enumerate}
    \item
      The symmetric functions~$e_1, e_2, \dotsc$ generate~$\Lambda$ and are algebraically independent.
    \item
      The monomials~$e_\lambda$ where~$\lambda$ ranges through all partitions form a~{\basis{$\Integer$}} of~$\Lambda$.
    \item
      The symmetric functions~$h_1, h_2, \dotsc$ generate~$\Lambda$ and are algebraically independent.
    \item
      The monomials~$h_\lambda$ where~$\lambda$ ranges through all partitions form a~{\basis{$\Integer$}} of~$\Lambda$.
    \item
      The symmetric functions~$s_\lambda$ where~$\lambda$ ranges through all partitions form a~{\basis{$\Integer$}} of~$\Lambda$.
  \end{enumerate}
\end{proposition}

% The mapping~$e_i(x_1, \dotsc, x_k) \mapsto e_i(x_1, \dotsc, x_{k+1})$ with~$i = 0, \dotsc, k$ extends to an embedding of rings~$\Lambda(k) \to \Lambda(k+1)$, and the above shows that one can regard~$\Lambda$ is the resulting colimit, i.e.\ as the polynomial ring~$\Integer[e_1, e_2, e_3, \dotsc]$.
% Similarly for~$h_i$ instead of~$e_i$.

\begin{theorem}
  Let~$\Phi \colon \Lambda \to R$ be the unique additive group homomorphism that maps the basis element~$e_\lambda$ to the element~$[\widetilde{M}^\lambda]$ where~$\lambda$ ranges through all partitions.
  \begin{enumerate}
    \item
      The map~$\Phi$ is an isomorphism of rings.
    \item
      It holds that~$\Phi(h_\lambda) = [M^\lambda]$.
    \item
      It holds that~$\Phi(s_\lambda) = [S^\lambda]$.
  \end{enumerate}
\end{theorem}

The multiplicity of~$\Phi$ stems from the identity~$\widetilde{M}^\lambda = \signrep_{\lambda_1} \circ \dotsb \circ \signrep_{\lambda_t}$ for~$\lambda = (\lambda_1, \dotsc, \lambda_t)$.





\section{Representations and symmetric polynomials}

\begin{corollary}
  The composition
  \[
    \Lambda
    \xlongto{\Phi^{-1}}
    R
    \xlongto{\Schur}
    \K_0(\GL(V))
    \xlongto{\Char}
    \Lambda(m)
  \]
  is given by~$f \mapsto f(x_1, \dotsc, x_m)$.
\end{corollary}

\begin{proof}
  The assertion holds for the basis elements~$s_\lambda$ of~$\Lambda$ as
  \[
    s_\lambda
    \mapsto
    [S^\lambda]
    \mapsto
    [\Schur(S^\lambda)]
    =
    [S^\lambda(V)]
    \mapsto
    s_\lambda(x_1, \dotsc, x_m) \,.
  \]
  The general assertion follow by additivity of all occuring maps.
\end{proof}

We have thus finally arrived at the following commutative diagram of rings:
\[
  \begin{tikzcd}[sep = huge]
    R
    \arrow[leftrightarrow]{r}{\sim}
    \arrow{d}[left]{\Schur}
    &
    \Lambda
    \arrow{d}[right]{f \mapsto f(x_1, \dotsc, x_m)}
    \\
    \K_0(\GL(V))
    \arrow{r}[above]{\sim}[below]{\Char}
    &
    \Lambda(m)
  \end{tikzcd}
\]
We have the following special cases of this diagram:
\begin{gather*}
  \begin{tikzcd}[ampersand replacement = \&]
    {[S^\lambda]}
    \arrow[leftrightarrow]{r}
    \arrow{d}
    \&
    s_\lambda
    \arrow{d}
    \\
    {[S^\lambda(V)]}
    \arrow[leftrightarrow]{r}
    \&
    s_\lambda(x_1, \dotsc, x_n)
  \end{tikzcd}
  \\
  \begin{tikzcd}[ampersand replacement = \&]
    {[\widetilde{M}^\lambda]}
    \arrow[leftrightarrow]{r}
    \arrow{d}
    \&
    e_\lambda
    \arrow{d}
    \\
    {[\widetilde{M}^\lambda(V)]}
    \arrow[leftrightarrow]{r}
    \&
    e_\lambda(x_1, \dotsc, x_n)
  \end{tikzcd}
  \qquad
  \begin{tikzcd}[ampersand replacement = \&]
    {[M^\lambda]}
    \arrow[leftrightarrow]{r}
    \arrow{d}
    \&
    h_\lambda
    \arrow{d}
    \\
    {[M^\lambda(V)]}
    \arrow[leftrightarrow]{r}
    \&
    h_\lambda(x_1, \dotsc, x_n)
  \end{tikzcd}
\end{gather*}

We can now use these correspondeces to translate between problems about the representation theory of~$\symm_n$, the representation theory of~$\GL(V)$, and the combinatorics of symmetric polynomials.

\begin{example}
  \leavevmode
  \begin{enumerate}
    \item
      For every~$n \geq 0$ there exists unique natural numbers~$f^\lambda$ for~$\lambda \ispar n$ such that one and thus all of the following conditions hold:
      \begin{enumerate}
        \item
          $\Complex[S_n] \cong \bigoplus_{\lambda \ispar n} (S^\lambda)^{\oplus f^\lambda}$,
        \item
          $V^{\tensor n} \cong \bigoplus_{\lambda \ispar n} S^\lambda(V)^{\oplus f^\lambda}$,
        \item
          $(x_1 + \dotsb + x_m)^n = \sum_{\lambda \ispar n} f^\lambda s_\lambda(x_1, \dotsc, x_m)$ for every~$m \geq 0$,
        \item
          $e_1^n = \sum_{\lambda \ispar n} f^\lambda s_\lambda$ in~$\Lambda$.
      \end{enumerate}
      We have already seen that it follows from the first description that~$f^\lambda$ is given by the number of standard Young tableaux of shape~$\lambda$.
    \item
      For every partition~$\lambda$ there exist unique natural numbers~$K_{\mu, \lambda}$ for~$\mu \strictlydominated \lambda$ such that one and thus all of the following conditions hold:
      \begin{enumerate}
        \item
          $M^\lambda \cong S^\lambda \oplus \bigoplus_{\mu \strictlydominated \lambda} (S^\mu)^{\oplus K_{\mu, \lambda}}$,
        \item
          $M^\lambda(V) \cong S^\lambda(V) \oplus \bigoplus_{\mu \strictlydominated \lambda} S^\mu(V)^{\oplus K_{\mu, \lambda}}$,
        \item
          $h_\lambda(x_1, \dotsc, x_m) = s_\lambda(x_1, \dotsc, x_m) + \sum_{\mu \strictlydominated \lambda} K_{\mu, \lambda} s_\mu(x_1, \dotsc, x_m)$ for all~$m \geq 0$,
        \item
          $h_\lambda = s_\lambda + \sum_{\mu \strictlydominated \lambda} K_{\mu, \lambda} s_\mu$.
      \end{enumerate}
      The numbers~$K_{\mu, \lambda}$ are the \defemph{Kostka numbers}.
    \item
      For every partition~$\lambda$ there exist unique natural numbers~$K_{\mu, \lambda}$ such that one and thus all of the following conditions hold:
      \begin{enumerate}
        \item
          $\widetilde{M}^\lambda \cong S^{\tilde{\lambda}} \oplus \bigoplus_{\tilde{\mu} \strictlydominated \lambda} (S^\mu)^{\oplus K_{\tilde{\mu}, \lambda}}$,
        \item
          $\widetilde{M}^\lambda(V) \cong S^{\tilde{\lambda}}(V) \oplus \bigoplus_{\tilde{\mu} \strictlydominated \lambda} S^\mu(V)^{\oplus K_{\tilde{\mu}, \lambda}}$,
        \item
          $e_\lambda(x_1, \dotsc, x_m) = s_{\tilde{\lambda}}(x_1, \dotsc, x_m) + \sum_{\tilde{\mu} \strictlydominated \lambda} K_{\tilde{\mu}, \lambda} s_\mu(x_1, \dotsc, x_m)$,
        \item
          $e_\lambda = s_{\tilde{\lambda}} + \sum_{\tilde{\mu} \strictlydominated \lambda} K_{\tilde{\mu}, \lambda} s_\mu$.
      \end{enumerate}
      The numbers~$K_{\mu, \lambda}$ are again the Kostka numbers as above.
    \item
      For any two partitions~$\lambda$ and~$\mu$ there exist natural numbers~$c^{\nu}_{\lambda, \mu}$ for~$\nu \ispar \size{\lambda} + \size{\mu}$ such that one and thus all of the following conditions hold:
      \begin{enumerate}
        \item
          $S^\lambda \circ S^\mu \cong \sum_\nu (S^\nu)^{\oplus c_{\lambda, \mu}^{\nu}}$,
        \item
          $S^\lambda(V) \tensor S^\mu(V) \cong \sum_\nu S^\nu(V)^{\oplus c_{\lambda, \mu}^{\nu}}$,
        \item
          $s_\lambda(x_1, \dotsc, x_m) s_\mu(x_1, \dotsc, x_m) = \sum_{\nu} c_{\lambda, \mu}^\nu s_\nu(x_1, \dotsc, x_m)$,
        \item
          $s_\lambda s_\mu = \sum_{\nu} c_{\lambda, \mu}^{\nu} s_\nu$.
      \end{enumerate}
      The numbers~$c_{\lambda, \mu}^{\nu}$ are the \defemph{Littlewood--Richardson coefficients}.
  \end{enumerate}
\end{example}

\end{document}
